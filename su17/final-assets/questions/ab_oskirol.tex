%!TEX root = ../final.tex
\question Arvind is testing the effectiveness of a drug, Oskirol, on patients
who suffer from the worldwide Tree Syndrome epidemic. He randomly selects 50
patients from the Junior University Hospital, then randomly assigns half of the
patients to a treatment group and half to control. He gives the treatment group
the drug and doesn't do anything to the control group.

Afterward, Arvind calculates an overall wellness score from 0 to 100 for all
patients. He creates a 95\% bootstrap confidence interval for the difference in
the mean wellness scores between treatment and control groups: [11.1,
18.4].

Sam has some complaints about Arvind's methods. Evaluate whether Sam is right
in each scenario and justify your answer.

\begin{parts}
  \part[3] Sam states that since Arvind only selected patients from the Junior
  University Hospital, Arvind cannot infer any effect his treatment has
  on \textbf{all} patients with Tree Syndrome.

  \begin{oneparcheckboxes}
    \correctchoice Sam is correct
    \choice Sam is incorrect
  \end{oneparcheckboxes}

  \begin{solutionorbox}[0.9in]
    Sam is correct. If Arvind only drew patients from the Junior University
    Hospital, with no other information he can only infer the effect his
    treatment has on patients with Tree Syndrome at that hospital.
  \end{solutionorbox}

  \part[3] Sam finds out that the true probability distribution of wellness
  scores is highly skewed left. Because of this, he states that the bootstrap
  is an \textbf{inappropriate method} to use.

  \begin{oneparcheckboxes}
    \choice Sam is correct
    \correctchoice Sam is incorrect
  \end{oneparcheckboxes}

  \begin{solutionorbox}[0.9in]
    Sam is incorrect. According to the Central Limit Theorem, the distribution
    of the mean wellness scores is normally distributed even though the actual
    distribution of scores is not, so the conditions for bootstrapping are
    still met.
  \end{solutionorbox}

  \part[3] Arvind notices that 0 is not in the confidence interval, so he
  concludes that is it likely his drug \textbf{caused} his patients' wellness
  scores to increase. Sam points out that Arvind's conclusion is incorrect:
  this hypothesis test can only be used to establish association, not
  causation.

  \begin{oneparcheckboxes}
    \choice Sam is correct
    \correctchoice Sam is incorrect
  \end{oneparcheckboxes}

  \begin{solutionorbox}[0.9in]
    Sam is incorrect. Because this experiment was a randomized control
    experiment, if we reject the null we can establish causality.

    Note that Arvind can conclude that his drug likely caused wellness scores
    to increase compared to patients who didn't take anything at all. He can't
    say anything about the effectiveness of his drug compared to a placebo
    which means this experiment needs adjusting for actual medical purposes.
  \end{solutionorbox}

  \part[3] Sam points out that Arvind didn't give a placebo to the
  control group, so the control group's wellness scores were not as high as
  they would be with a placebo. Sam argues that Arvind's confidence interval
  endpoints are likely \textbf{lower} than they would be with the placebo.

  \begin{oneparcheckboxes}
    \choice Sam is correct
    \correctchoice Sam is incorrect
  \end{oneparcheckboxes}

  \begin{solutionorbox}[0.9in]
    Sam is incorrect. Since we expect a placebo to increase the wellness scores
    of the control group, Arvind's confidence interval endpoints are likely
    higher that the one he would have created if he gave a placebo to the
    control group.
  \end{solutionorbox}

\end{parts}
