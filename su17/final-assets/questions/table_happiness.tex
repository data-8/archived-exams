%!TEX root = ../final.tex
\question The \code{happy} table from Gallup World Poll data describes the
happiness score ratings of 157 countries. It contains the number of people (in
thousands) that gave each happiness rating (0 is worst, 10 is best) for each
country.

\begin{verbatim}
Country | Rating | Count
Denmark | 0      | 3.4
Denmark | 1      | 4.5
...     | ...    | ...
Denmark | 10     | 16.9
Iceland | 0      | 2.1
Iceland | 1      | 4.1
... (1510 rows omitted)
\end{verbatim}

Complete the Python expressions below to compute each result. The last line of
each answer should evaluate to the result requested. \textbf{You may not use
more lines than the ones provided.}

\begin{parts}

\part[2] The total number of people surveyed in the data, in thousands.

\begin{verbatim}
np.sum(_________________________________________________________)
\end{verbatim}

\begin{solution}
\begin{verbatim}
  np.sum(happy.column('Count'))
\end{verbatim}
\end{solution}

\part[3] A table containing the same columns as \code{happy} with the countries
in descending order by the number of people in that country who gave a rating
of 10. (All ratings in the table are integers.)

\begin{verbatim}
happy.__________________________________________________________
\end{verbatim}

\begin{solution}
\begin{verbatim}
happy.where('Rating', 10).sort('Count', descending=True)
\end{verbatim}
\end{solution}

\part[3] A table containing two columns: the countries and the total amount of
people surveyed for each country in thousands.
\begin{verbatim}
happy._________________________________________________________
\end{verbatim}

\begin{solution}
\begin{verbatim}
happy.drop('Rating').group('Country', np.sum)
\end{verbatim}
\end{solution}

\part[3] The average happiness rating of Denmark. (The answer is not simply 5.
Remember to account for the rating counts.)
\begin{verbatim}
d = ____________________________________________________________

sum(________________ * __________________) / sum(______________)
\end{verbatim}

\begin{solution}
\begin{verbatim}
d = happy.where('Country', 'Denmark')
sum(d.column(1) * d.column(2)) / sum(d.column(2))
\end{verbatim}
\end{solution}

\part[6] A table containing the average happiness rating for each country.
Assume you have a table called \code{totals} that contains the total
people surveyed for each country:

\begin{verbatim}
Country | Total
Denmark | 41.4
Iceland | 38.9
... (155 rows omitted)
\end{verbatim}

\begin{verbatim}
def helper(row):
    total = totals.____________________________________________

    return _________________ * ________________ / _____________

(happy.with_column('score', ___________________________________)
      .select('Country', 'score')
      .group('Country', np.sum))
\end{verbatim}

\begin{solution}
\begin{verbatim}
def helper(row):
    total = totals.where('Country', row.item(0)).column(1).item(0)
    return row.item(1) * row.item(2) / total

(happy.with_column('score', happy.apply(helper))
      .select('Country', 'score')
      .group('Country', np.sum))
\end{verbatim}
\end{solution}

\end{parts}
