\q{8}{P-Value Puzzle}


Define the function {\tt p\_value\_calculation}, which takes in the following 3 arguments: 

\begin{itemize}
\item {\tt sim\_vals} is an array of simulated test statistics under a specific null hypothesis.
\item {\tt observed\_ts} is the value of a test statistic from a specific observed sample.
\item {\tt larger\_alt} is a boolean which is either {\tt True} if only large values of the test statistic above point towards the alternative, and {\tt False} otherwise (small values of the test statistic point towards the alternative).
\end{itemize} 
The function should calculate the P-Value of the {\tt observed\_ts} (observed statistic), with respect to the null hypothesis under which {\tt sim\_vals} was simulated. You may not need all of the lines. 
\\ \\
\lstinline{def p_value_calculation(sim_vals, observed_ts, larger_alt):} \\


\lstinline{	_______________________________________________________________________________} \\ \\


\lstinline{	_______________________________________________________________________________} \\ \\


\lstinline{	_______________________________________________________________________________}\\ \\

\lstinline{        ________________________________________________________________________________} \\ \\


\solution{
\lstinline{def p_value_calculation(sim_vals, observed_ts, larger_alt):} \\
\lstinline{	if larger_alt:} \\ \\
\lstinline{	    return sum(sim_vals >= observed_ts) / len(sim_vals)} \\ \\
\lstinline{	else:}\\ \\
\lstinline{     return sum(sim_vals <= observed_ts) / len(sim_vals)}\
}


