\q{8}{Street Smarts}
\vskip .1in
In 2015, economists Melissa Kearney and Phillip Levine published a study investigating the effects of watching \textit{Sesame Street} on the future academic performance of young viewers in the early 1970s. \textit{Sesame Street} is a popular children’s show that was first introduced in 1969 and provides free educational content on public television to children who are too young to attend school.
\vskip .2in
However, there was variation in exposure to \textit{Sesame Street} across the country, due to technological constraints. Approximately 1/3 of US viewers lived in counties where they were unlikely to be able to view \textit{Sesame Street}. Whether or not a county had high rates of access to \textit{Sesame Street} was close to random and was not associated with any particular characteristics of the county, such as income.
\vskip .2in
Since the researchers did not have data on whether individual children watched \textit{Sesame Street}, they instead investigated county-level elementary-school success metrics of groups of children who started school after the show debuted and who lived in locations where broadcast reception for the show was high, and compared these metrics to those among older groups of children (who would not have watched \textit{Sesame Street} before starting school) and those who lived in locations with limited broadcast reception (poor likelihood of access to \textit{Sesame Street}). 
\vskip .2in
The authors found that "children who lived in places with better access to the show did better [on average] in elementary school, as compared to those with limited access and those who were older at the time the show was introduced. They were more likely to start school on time and progress at the appropriate grade for age."
\begin{enumerate}
\vskip .3in
\subq{2} Who were the individuals in this study?\vskip 0.1in


{\bubble} Individual children who started school after \textit{Sesame Street} debuted in the United States\\[2pt]
{\bubble} Individual children who started school just before or after \textit{Sesame Street} debuted in the United States\\[2pt]
\solutionimage{\bubble}{\filledbubble}  Groups of children in different counties in the United States\\[2pt]
{\bubble}  Groups of children in different counties with high likelihood of access to \textit{Sesame Street} in the United States


\vskip .25in
\subq{2} The authors made two comparisons:\vskip 0.1in
\begin{itemize}
\item They compared the educational outcomes of groups of children in counties with high likelihood of access to \textit{Sesame Street} to the educational outcomes of groups of children with low likelihood of access to \textit{Sesame Street}.
\item They compared the educational outcomes of groups of children in counties with high access to \textit{Sesame Street}, and who started school after \textit{Sesame Street} debuted, to groups of children \textbf{in those same high-access counties} who started school before \textit{Sesame Street} debuted on television.
\end{itemize}
\vskip 0.1in
Why did the authors include both comparisons in their study?

{\bubble} The authors thought children may have moved from one county to another, so they wanted to study the children who moved by looking at the same group over time.\\[4pt]
{\bubble} By studying the same children before and after they watched \textit{Sesame Street}, the authors could determine whether \textit{Sesame Street} was the reason that some individual students performed better than others.\\[4pt]
{\bubble}  The authors wanted to include more children in their study (a larger sample size) so that the children would be more representative of the population of all children in the United States, instead of just looking at one age group.\\[4pt]
\solutionimage{\bubble}{\filledbubble}  Making both comparisons helps eliminate the possibility of confounding factors due to broadcast variability (which counties had \textit{Sesame Street} access) or variation in educational quality across years (if education in general improved or got worse over the study period).

\vskip .4in
\subq{4} Select \textbf{all} correct statements:\vskip 0.1in


\solutionimage{\bubble}{\filledbubble} This was an observational study.\\[2pt]
{\bubble} This was a randomized controlled experiment.\\[2pt]
{\bubble} It is only possible to investigate a causal link with a randomized controlled experiment, and this study did not help establish a causal link between \textit{Sesame Street} and better educational outcomes.\\[2pt]
{\bubble} Whether or not a child lived somewhere with access to \textit{Sesame Street} was random, \\because the researchers split the groups of children into a treatment and a control group.\\[2pt]
\solutionimage{\bubble}{\filledbubble}  The authors were attempting to use a natural experiment, the variation in TV signals, \\to establish a causal link between watching \textit{Sesame Street} and better educational outcomes.\\[2pt]


\end{enumerate}